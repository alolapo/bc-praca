\documentclass[12pt, oneside]{book}
\usepackage[a4paper,top=2.5cm,bottom=2.5cm,left=3.5cm,right=2cm]{geometry}
\usepackage[utf8]{inputenc}
\usepackage[T1]{fontenc}
\usepackage{graphicx}
\usepackage{url}
\usepackage[hidelinks,breaklinks]{hyperref}
\usepackage[slovak]{babel} % vypnite pre prace v anglictine
\usepackage{color}

\usepackage{listings}
\usepackage{verbatim}

\linespread{1.25} % hodnota 1.25 by mala zodpovedat 1.5 riadkovaniu

\definecolor{lightgray}{rgb}{.9,.9,.9}
\definecolor{darkgray}{rgb}{.4,.4,.4}
\definecolor{purple}{rgb}{0.65, 0.12, 0.82}

\lstdefinelanguage{JavaScript}{
  keywords={typeof, new, true, false, then, catch, function, return, null, catch, switch, var, let, const, get, if, in, while, do, else, case, break},
  keywordstyle=\color{blue}\bfseries,
  ndkeywords={class, export, boolean, throw, implements, import, this},
  ndkeywordstyle=\color{darkgray}\bfseries,
  identifierstyle=\color{black},
  sensitive=false,
  comment=[l]{//},
  morecomment=[s]{/*}{*/},
  commentstyle=\color{purple}\ttfamily,
  stringstyle=\color{red}\ttfamily,
  morestring=[b]',
  morestring=[b]"
}

\lstset{
   language=JavaScript,
   %backgroundcolor=\color{lightgray},
   frame=single,
   extendedchars=true,
   basicstyle=\scriptsize\ttfamily,
   showstringspaces=false,
   showspaces=false,
   numbers=left,
   numberstyle=\scriptsize,
   numbersep=9pt,
   tabsize=2,
   breaklines=true,
   showtabs=false,
   captionpos=b
}

% -------------------
% --- Definicia zakladnych pojmov
% --- Vyplnte podla vasho zadania
% -------------------
\def\JS{{ECMAScript® 2016}}%TODO preco to oddeluje 2016?
%\def\TODO{{\textcolor{red}{\textbf{TODO} }}}
%\def\NEW{{\textcolor{green}{\textbf{NEW} }}}

\def\mfrok{2017}
\def\mfnazov{Porovnanie návrhových vzorov Flux v jazyku Dart a Redux v \JS{} a ich vplyv na vývoj single-page aplikácie}
\def\mftyp{Bakalárska práca}
\def\mfautor{Alena Poláchová}
\def\mfskolitel{Mgr. Jakub Uhrík}

%ak mate konzultanta, odkomentujte aj jeho meno na titulnom liste
\def\mfkonzultant{tit. Meno Priezvisko, tit. }  

\def\mfmiesto{Bratislava, \mfrok}

%aj cislo odboru je povinne a je podla studijneho odboru autora prace
\def\mfodbor{2508 Informatika} 
\def\program{ Informatika }
\def\mfpracovisko{ Katedra informatiky }% alebo BookYourself, s.r.o.

\begin{document}     
\frontmatter


% -------------------
% --- Obalka ------
% -------------------
\thispagestyle{empty}

\begin{center}
\sc\large
Univerzita Komenského v Bratislave\\
Fakulta matematiky, fyziky a informatiky

\vfill

{\LARGE\mfnazov}\\
\mftyp
\end{center}

\vfill

{\sc\large 
\noindent \mfrok\\
\mfautor
}

\eject % EOP i
% --- koniec obalky ----

% -------------------
% --- Titulný list
% -------------------

\thispagestyle{empty}
\noindent

\begin{center}
\sc  
\large
Univerzita Komenského v Bratislave\\
Fakulta matematiky, fyziky a informatiky

\vfill

{\LARGE\mfnazov}\\
\mftyp
\end{center}

\vfill

\noindent
\begin{tabular}{ll}
Študijný program: & \program \\
Študijný odbor: & \mfodbor \\
Školiace pracovisko: & \mfpracovisko \\
Školiteľ: & \mfskolitel \\
% Konzultant: & \mfkonzultant \\
\end{tabular}

\vfill


\noindent \mfmiesto\\
\mfautor

\eject % EOP i


% --- Koniec titulnej strany


% -------------------
% --- Zadanie z AIS
% -------------------
% v tlačenej verzii s podpismi zainteresovaných osôb.
% v elektronickej verzii sa zverejňuje zadanie bez podpisov

\newpage 
\thispagestyle{empty}
\hspace{-2cm}\includegraphics[width=1.1\textwidth]{images/zadanie}

% --- Koniec zadania

\frontmatter

% -------------------
%   Poďakovanie - nepovinné
% -------------------
\setcounter{page}{3}
\newpage 
~

\vfill
{\bf Poďakovanie:} Chcela by som sa poďakovať môjmu školiteľovi Mgr. Jakubovi Uhríkovi za jeho trpezlivosť s mojimi otázkami, ochotu pripomienkovať všetky verzie práce, podporu a nadšenie, ktoré mi dodávalo nádej, že to všetko dopadne dobre :). %Ďalej by som sa chcela poďakovať kamarátovi Damimu, ktorý mi pomohol opraviť vyčerpávajúci zoznam gramatických chýb. A v neposlednom rade by som sa chcela poďakovať mojej rodine, ktorá ma po celý ten čas podporovala.

% --- Koniec poďakovania

% -------------------
%   Abstrakt - Slovensky
% -------------------
\newpage 
\section*{Abstrakt}
V práci porovnávame single-page aplikáciu v jazyku Dart s použitím návrhového vzoru Flux 
s aplikáciou v jazyku \JS{} s návrhovým vzorom Redux.
Popisujeme, ako by sa dala aplikácia z jazyka Dart so vzorom Flux 
efektívne presunúť do jazyka \JS{} s návrhovým vzorom Redux. 
V prospech efektivity sa snažíme zachovať maximum z pôvodného kódu.
Vysvetľujeme motiváciu, prečo je tento krok prínosný pri vývoji single-page aplikácie.

\paragraph*{Kľúčové slová:} Dart, \JS{}, Flux, Redux, SPA
% --- Koniec Abstrakt - Slovensky


% -------------------
% --- Abstrakt - Anglicky 
% -------------------
\newpage 
\section*{Abstract}

Thesis compares a single-page application in Dart language with design pattern Flux to an application in \JS{} language with design pattern Redux.
We describe, how application in the Dart language with pattern Flux could effectively be moved to the \JS{} language with pattern Redux.
In behalf of efficiency we try to reuse the most of the original code.
We explain motivation, why this step is beneficial in development of single-page application.


\paragraph*{Keywords:} Dart, \JS{}, Flux, Redux, SPA

% --- Koniec Abstrakt - Anglicky

% -------------------
% --- Predhovor - v informatike sa zvacsa nepouziva
% -------------------
%\newpage 
%\thispagestyle{empty}
%
%\huge{Predhovor}
%\normalsize
%\newline
%Predhovor je všeobecná informácia o práci, obsahuje hlavnú charakteristiku práce 
%a okolnosti jej vzniku. Autor zdôvodní výber témy, stručne informuje o cieľoch 
%a význame práce, spomenie domáci a zahraničný kontext, komu je práca určená, 
%použité metódy, stav poznania; autor stručne charakterizuje svoj prístup a svoje 
%hľadisko. 
%
% --- Koniec Predhovor


% -------------------
% --- Obsah
% -------------------

\newpage 

\tableofcontents

% ---  Koniec Obsahu

% -------------------
% --- Zoznamy tabuliek, obrázkov - nepovinne
% -------------------

%\newpage 

%\listoffigures
%\listoftables

% ---  Koniec Zoznamov

\mainmatter


\input uvod.tex 

\input prostredie.tex

\input syntax.tex

\input logika.tex

\input aplikacia.tex

%\input kapitola.tex

%\input latex.tex

\input zaver.tex

% -------------------
% --- Bibliografia
% -------------------


\newpage	

\backmatter

\thispagestyle{empty}
\nocite{*}
\clearpage

\bibliographystyle{plain}
\bibliography{literatura} 

%Prípadne môžete napísať literatúru priamo tu
%\begin{thebibliography}{5}
 
%\bibitem{br1} MOLINA H. G. - ULLMAN J. D. - WIDOM J., 2002, Database Systems, Upper Saddle River : Prentice-Hall, 2002, 1119 s., Pearson International edition, 0-13-098043-9

%\bibitem{br2} MOLINA H. G. - ULLMAN J. D. - WIDOM J., 2000 , Databasse System implementation, New Jersey : Prentice-Hall, 2000, 653s., ???

%\bibitem{br3} ULLMAN J. D. - WIDOM J., 1997, A First Course in Database Systems, New Jersey : Prentice-Hall, 1997, 470s., 

%\bibitem{br4} PREFUSE, 2007, The Prefuse visualization toolkit,  [online] Dostupné na internete: <http://prefuse.org/>

%\bibitem{br5} PREFUSE Forum, Sourceforge - Prefuse Forum,  [online] Dostupné na internete: <http://sourceforge.net/projects/prefuse/>

%\end{thebibliography}

%---koniec Referencii

% -------------------
%--- Prilohy---
% -------------------

%Nepovinná časť prílohy obsahuje materiály, ktoré neboli zaradené priamo  do textu. Každá príloha sa začína na novej strane.
%Zoznam príloh je súčasťou obsahu.
%
%\addcontentsline{toc}{chapter}{Appendix A}
%\input AppendixA.tex
%
%\addcontentsline{toc}{chapter}{Appendix B}
%\input AppendixB.tex

\end{document}






