\chapter*{Záver}  % chapter* je necislovana kapitola
\addcontentsline{toc}{chapter}{Záver} % rucne pridanie do obsahu
\markboth{Záver}{Záver} % vyriesenie hlaviciek

Programovacie jazyky Dart a \JS{} sú založené na podobných princípoch. Majú však niekoľko syntaktických aj logických odlišností, ktoré sme využili pri preklade kódu v náš prospech. Snažili sme sa zamerať na udržateľnosť kódu a preto sme zvolili možnosť využiť čo najviac existujúcich knižníc.

Na intenete existuje viacero návodov ako preložiť aplikáciu z návrhového vzoru Flux do vzoru Redux. Úlohou tejto bakalárskej práce bolo pochopiť tieto návrhové vzory a dať názorný príklad na ukážke kódu.

V práci sme tiež uviedli tabuľky, ktoré by mali slúžiť ako návod pri písaní aplikácie so zameraním na daný návrhový vzor.

%V závere mojej bakalárskej práce zhrniem, aké bolo zadanie, ako som postupovala a ako by sa dalo na moju prácu nadviazať.
