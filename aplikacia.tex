\chapter{Aplikácia}
Téma na túto prácu vznikla z praxe. 

\paragraph{Zmena programovacieho jazyka}
Existujúca aplikácia vyvíjaná v jazyku Dart už má svoje zabehnuté prostredie v ktorom beží. Dart je výborný jazyk pre začínajúci tím (má voliteľnú kontrolu typov, pomerne presnú štruktúru.) Už však nie je taký živý a knižnice ku nemu vznikajú pomaly a v obmedzenom množstve. 
Taktiež podpora vývojových prostredí je menšia a slabšia. (V čase, keď bežal jazyk Dart naplno, existovalo preň vývojové prostredie, dnes existuje viacero programov s podporou Dart-u.) 
Ďalším dôvodom bol fakt, že JavaScript beží v prehliadači priamo, zatiaľ čo Dart bolo potrebné v produkcii kompilovať do JavaScript-u, keďže nie je podporovaný všetkými prehliadačmi.

Taktiež bolo potrebné písať si vlastné knižnice na zobrazovanie komponentov a ich štýlovanie.

Preto bola snaha presunúť už existujúci kód do jazyka JavaScript, ktorý je v súčastnosti veľmi živý jazyk podporovaný mnohými komunitami. Pri vývoji aplikácie v JavaScripte sme sa zamerali na lean vývoj (/fail fast) preto sme si ako základ aplikácie zvolili balík knižníc \cite[este]{Este}.

Napriek tomu tieto jazyky sú navzájom veľmi podobné a bolo jednoduché prispôsobiť sa jazyku JavaScript po skúsenostiach s Dartom.
Rozdiel bol viac v použitých knižniciach ako v samotných jazykoch.

Pri použití knižnice React som zaznamenala nevýhodu, keď nebolo možné používať syntax pre jeden komponent na pole viacerých komponentov, čo som vyriešila obalením poľa pomocným komponentom. Tiež by sa tento problém dal riešiť použitím funkcie namiesto syntaxe pre komponent.

\begin{lstlisting}[caption=Pole komponentov v Darte]
  get buttons => [
    proforma,
    packed,
    cancelOrder,
    paid,
  ];
\end{lstlisting}

\begin{lstlisting}[caption=Pole komponentov v JavaScripte s použitím knižnice Redux]
const Buttons = order => (
  <Box>
    <Proforma props={{ order }} />
    <Packed props={{ order }} />
    <CancelOrder props={{ order }} />
    <Paid props={{ order }} />
  </Box>
);
\end{lstlisting}

\paragraph{Jazykové mutácie}
Veľký prínos som zaznamenala v spravovaní jazykových mutácií aplikácie. Knižnica \emph{react-intl} vie vyexportovať nepreložené a nové frázy, aj keď nie sú uložené v jednom súbore, čo sprehľadňuje správu týchto fráz.

\paragraph{Zmena návrhového vzoru}
Flux je pomerne jednoduchý návrhový vzor. Výhodou Reduxu oproti fluxu je však ešte viac prehladná aplikácia. Pri Reduxe je jasné, čo a kedy sa zmenilo. Redux vďaka čistým funkciám a zoznamu vykonaných akcií poskytuje možnosť simulovať celý beh v ľubovolnom čase, zopakovať niektoré kroky s rovnakým stavom aplikácie alebo sa vrátiť v čase dozadu.

V pôvodnej aplikácii sme robili dotazy na server z jednotlivých storov. V novej aplikácii sme toto nemohli nechať na reducer, pretože by to porušilo princípy Redux aplikácie. Preto tieto dotazy robí middleware. Pri písaní novej aplikácie ma tento vzor naučil zamýšľať sa nad tým, ktorá akcia je príčinou takéhoto dotazu na server, pretože bolo potrebné počúvať práve na ňu.

\begin{comment}

\begin{lstlisting}[caption=Dotaz na server vo Flux-e v store]
//zmena v stave - store
	listen({
		...
    "$ORDER-$PACKED": _markOrderAs(ORDERPACKED),
    ...
  });

  ...

  _markOrderAs(String state) => (event) {
    insert([ORDER, STATUS], state);

    dispatch("$RETAILER-$ORDER-$EDIT-$STATUS",
        data: getInOr(event, DATA, per({}))
    );
  };

//zmena poslana na server - store
	listen({
		...
		"$RETAILER-$ORDER-$EDIT-$STATUS": _updateOrderStatus,
		...
	});

  ...

  _updateOrderStatus(event) {

    var eventData = getMap(event, DATA);

    eventData = insertInOrCreate(eventData, STATUS, getInOr(data, [ORDER, STATUS]));

    String id = getInOr(data, [ORDER, ID]);

    DiscoveryMap aditionalData =  new DiscoveryMap.fromJson(unpersist(eventData));

    insert([ORDER], into(getInOr(data,[ORDER]), eventData));

    dispatchAsync("$RETAILER-$ORDER-$UPDATED", resource.patch(aditionalData, id));

  }
\end{lstlisting}

\begin{lstlisting}[caption=Dotaz na server v Redux-e cez middleware]
//zmena v stave - reducer
  switch (action.type) {
  	...
    case "CHANGE_ENTITY_FIELD": {
      var {entity, id, field, value} = action.payload;

      var schema = schemas(entity);

      var newField = {}
      newField[field] = value;

      var oldEntity = denormalize(id, schema, state);
      var newEntity = {...oldEntity, ...newField};
      var normalized = normalize(newEntity, schema);

      var normalizedEntities = {...state[entity], [id]: normalized.entities[entity][id]};

      return {...state, [entity]: normalizedEntities};
    }

    default:
      return state;

  }

//zmena poslana na server - middleware
const entityChangedEpic = (action$: any, { firebase }: Deps) =>
  action$
    .filter((action: Action) => action.type === "CHANGE_ENTITY_FIELD")
    .mergeMap(action => {
      var { entity, id, field, value } = action.payload;

      var newField = {};
      newField[field] = value;
      var source = `${pl(entity)}/${id}.json`
      const promise = fetch(serverUrl + source, {
        method: "PATCH",
        headers: new Headers({
          "X-Requested-With": "eusahub-redux-app",
          "X-USER-TOKEN": token,
          "Content-Type": "application/json"
        }),
        body: JSON.stringify(newField)
      });

      return Observable.from(promise.then(response => response.json()))
        .mergeMap(response =>
          Observable.from(entityPatched(entity, response)))
        .catch(error => Observable.of(appError(error)));
    });
\end{lstlisting}

\end{comment}

\paragraph{Entity}%TODO vhodne umiestniť
V aplikácii sme spravili jednu zásadnú zmenu so štruktúrou vnútorného stavu aplikácie. Doteraz boli entity (objednávky, protokoly) priamo v stave pod daným kľúčom. 
V Reduxovej aplikácii sme ich presunuli do samostatnej vetvy, kde sú jednotlivé entity uložené pod typom a identifikačným reťazcom. Potom v stave namiesto zoznamu celých entít existuje len zoznam ID. 
Túto zmenu sme spravili preto, aby bolo jednoduchšie upravovať jednotlivé entity. Teda aj keď sa na danú entitu odkazujeme na viacerých miestach, jej dáta upravujeme len na jednom. Túto funkcionalitu nám pomáha udržiavať knižnica \emph{normalizr}.
Tiež pri veľkom množstve entít to uľahčilo vyhľadávanie danej entity podľa ID (netreba iterovať cez pole entít, stačí sa pozrieť do mapy pod správnym kľúčom).