\chapter*{Úvod} % chapter* je necislovana kapitola
\addcontentsline{toc}{chapter}{Úvod} % rucne pridanie do obsahu
\markboth{Úvod}{Úvod} % vyriesenie hlaviciek

Práca prevedie čitateľa pomerne jednoduchým návodom, ako zmigrovať aplikáciu 
z~jazyka Dart do jazyka \JS{} s ohľadom na návrhové vzory Flux a Redux. 
V úvode poslednej kapitoly vysvetľujeme našu motiváciu, pre ktorú preferujeme kód 
v~\JS{} s návrhovým vzorom Redux. Práca odzrkadľuje reálnu potrebu z~praxe.

V prvej kapitole stručne uvedieme, akej časti informatiky sa venujeme. Popíšeme single-page aplikáciu ako základ našej práce, v ktorej sa budeme celý čas pohybovať.

V kapitole \ref{kap:jazyky} popisujeme programovacie jazyky Dart a \JS{}, vymenúvame a stručne vysvetľujeme ich vlastnosti. Na konci kapitoly uvádzame porovnanie hlavných čŕt týchto programovacích jazykov v tabuľke.

Kapitola \ref{kap:vzory} je venovaná návrhovým vzorom Flux a Redux. Zameriavame sa na hlavné časti týchto návrhových vzorov a ich funkciu. Stručne načrtneme, prečo preferujeme návrhový vzor Redux.

V kapitole \ref{kap:motivacia} v úvode uvádzame, prečo v praxi preferujeme jazyk Dart pred jazykom \JS{}. Rovnako popisujeme, prečo sme si vybrali presunúť aplikáciu zo vzoru Flux do vzoru Redux.

V druhej časti kapitoly \ref{kap:motivacia} sa venujeme samotnému návrhu migrácie jednotlivých častí vzoru Flux do vzoru Redux s ohľadom na programovacie jazyky. Porovnávame tieto časti v tabuľkách, ktoré môžu slúžiť ako návody pri migrácii takejto aplikácie. V tejto kapitole tiež prikladáme ukážky kódu, ako by mohla vyzerať pôvodná a zmigrovaná aplikácia. Venujeme sa podrobnejšie niektorým špecifikám jazyka Dart.

V ďalšej časti kapitoly \ref{kap:motivacia} si predstavíme knižnice, ktoré sme použili my pri spomínanej migrácii.

V závere kapitoly \ref{kap:motivacia} si povieme o súborovej štruktúre oboch aplikácií.
