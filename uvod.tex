\chapter*{Úvod} % chapter* je necislovana kapitola
\addcontentsline{toc}{chapter}{Úvod} % rucne pridanie do obsahu
\markboth{Úvod}{Úvod} % vyriesenie hlaviciek

%Cieľom tejto práce je...
%Pri vývoji single-page aplikácie v jazyku Dart s návrhovým vzorom Flux sme sa stretli
TODO{} (nejaká motivácia, že sme robili v Darte a všimli sme si, že JavaScript je viac podporovaný a...)

Práca by mala previesť čitateľa pomerne jednoduchým návodom, ako zmigrovať aplikáciu z jazyka Dart do jazyka \JS{} s ohľadom na návrhové vzory Flux a Redux. V úvode kapitoly \ref{kap:motivacia} uvádzame motiváciu, prečo preferujeme kód v \JS{} s návrhovým vzorom Redux.

v kapitole \ref{kap:prostredie} stručne uvedieme, akej časti informatiky sa venujeme. Popíšeme single-page aplikáciu ako základ našej práce, v ktorej sa budeme celý čas pohybovať.

V kapitole \ref{kap:jazyky} popisujeme programovacie jazyky Dart a \JS{}, vymenúvame a stručne vysvetľujeme ich vlastnosti. Na konci kapitoly uvádzame porovnanie hlavných čŕt týchto programovacích jazykov v tabuľke.

V kapitole \ref{kap:vzory} popisujeme návrhové vzory Flux a Redux. Zameriavame sa na hlavné časti týchto návrhových vzorov a na ich funkciu v týchto návrhových vzoroch. Neskôr v kapitole porovnávame tieto časti vzorov. Stručne načrtneme, prečo preferujeme návrhový vzor Redux.

V kapitole \ref{kap:motivacia} v úvode popisujeme, prečo preferujeme jazyk Dart pred jazykom \JS{}. Rovnako popisujeme, prečo sme si vybrali presunúť aplikáciu zo vzoru Flux do vzoru Redux.

V ďalšej časti kapitoly \ref{kap:motivacia} sa venujeme návrhu migrácie jednotlivých častí vzoru Flux do vzoru Redux s ohľadom na programovacie jazyky. Porovnávame tieto časti v tabuľkách, ktoré môžu slúžiť ako návody pri migrácii takejto aplikácie. V tejto kapitole tiež prikladáme ukážky kódu, ako by mohla vyzerať pôvodná a zmigrovaná aplikácia. Venujeme sa podrobnejšie niektorým špeciálnym() vlastnostiam jazyka Dart.

V časti \ref{sec:kniznice} kapitoly \ref{kap:motivacia} si predstavíme knižnice, ktoré sme použili my pri spomínanej migrácii.

V závere kapitoly \ref{kap:motivacia} si povieme o súborovej štruktúre oboch aplikácií.

\begin{comment}

Tu bude úvod do problematiky o mojej bakalárskej práci, načrtnutie problému a stručné popísanie obsahu jednotlivých kapitol.
\end{comment}