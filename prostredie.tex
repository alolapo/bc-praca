\chapter{Prostredie}

\label{kap:prostredie} % id kapitoly pre prikaz ref

V tejto kapitole si predstavíme vstupný popis kódu, ktorý sa budeme snažiť podľa zadaných podmienok zmeniť.

Pôvodná aplikácia je napísaná v jazyku Dart, ktorý bližšie popíšeme v podkapitole \ref{sec:dart}. Je to internetová aplikácia bežiaca v prehliadači založená na princípoch SPA-aplikácie.

\section{Single-Page Application (SPA)}
SPA, teda aplikácia fungujúca na jedno načítanie je model internetovej aplikácie. Ponúka rýchlosť desktopovej aplikácie a zároveň dostupnosť internetovej stránky.

Celá stránka je načítaná len raz, a to na začiatku. Logika aplikácie je potom kontrolovaná skriptom v prehliadači na strane klienta. 
Komunikácia so serverom prebieha len v malom množstve prípadov, ako je napríklad validácia údajov, autentifikácia alebo dostupnosť zdieľaných dát. 
Taktiež v čase, keď klient komunikuje so serverom, je možné zobraziť používateľovi vhodnú hlášku o spracovaní dát (na rozdiel od aplikácií, kde je stránka generovaná na serveri a zobrazí sa klientovi až po úplnom načítaní).

V takýchto aplikáciách sa často využíva práve jazyk JavaScript. Veľkou výhodou je multiplatformová dostupnosť vďaka internetovým prehliadačom a bez nutnosti inštalácie ďalších podporných programov. Podrobnejší popis sa dá nájsť v manuáli o SPA \cite{SPA}.

%- pri zmene obsahu sa prekresľuje len zmenená časť (DOM-elementy, aplikácia si pamätá aktuálny vykreslený stav stránky a pri zmene nejakej časti stránky netreba prekresľovať nezmenené prvky.)

%- stav uložený v URL(Časť stavu aplikácie sa môže ukladať v url...??)
