\chapter{Programovacie jazyky Dart a \JS}

\label{kap:jazyky} % id kapitoly pre prikaz ref

V tejto kapitole si povieme niečo o programovacích jazykoch Dart a \JS{}, s ktorými budeme počas celej práce robiť.

\section{Dart}
\label{sec:dart}
V tejto sekcii si predstavíme hlavné črty programovacieho jazyka Dart. Dart je objektovo orientovaný programovací jazyk. Je založený na definovaní tried, kde trieda môže dediť od najviac jednej inej triedy. 
Jazyk Dart je voliteľne typovaný. (Túto vlastnosť si bližšie popíšeme v podkapitole \ref{subsec:typy}). 
%(+ supports reified generics?).
Informácie do tejto kapitoly boli čerpané najmä zo špecifikácie jazyka\cite{DartLanguage}.

\subsection{Triedy}
Trieda (\emph{class}) definuje formu a správanie určitej množine objektov. Tieto objekty nazývame inštancie (\emph{instance}) danej triedy. 
Trieda môže byť definovaná deklaráciou samotnej triedy, alebo pomocou mixinov.

Trieda má konštruktory a členy (členy danej inštancie a statické členy). Členy sú metódy, premenné, gettery a settery. 

%superclass
\paragraph{Nadtrieda}
Každá trieda má práve jednu nadtriedu (\emph{superclass}) okrem triedy \emph{Object}, ktorá nemá. Táto nadtrieda sa uvádza za kľúčovým slovom \emph{extends}, alebo je určená implicitne ako \emph{Object}. %with - mixin
Daná trieda dedí od nadtriedy všetky dostupné členy danej inštancie, ktoré neboli preťažené (\emph{override}).

%interface
\paragraph{Rozhranie}
Trieda môže implementovať (\emph{implements}) niekoľko rozhraní (\emph{interface}). Rozhranie definuje, ako by sa malo pracovať s objektom. 
Má metódy, gettery, settery a množinu "nadrozhraní", ktoré rozširuje.

%Mixins
\paragraph{Mixin}
\label{par:mixins}
Mixin opisuje rozdiel medzi triedou a jej nadtriedou. Mixin je vždy odvodený od deklarácie existujúcej triedy.
Mixin môžeme použiť pri definovaní novej triedy pomocou kľúčového slova \emph{with M} (kde M je mixin).
Mixin je užitočný v prípadoch, kedy viacerým zdanlivo nezávislým triedam chceme pridať rovnakú funkcionalitu.

%abstract
\paragraph{Abstraktná trieda}
Trieda môže byť abstraktná, vtedy ju definujeme kľúčovým slovom \emph{abstract}. Takáto trieda nemusí mať implementované všetky metódy. Mixiny sú abstraktné triedy.

%constructor
\paragraph{Konštruktor}
Konštruktor triedy je špeciálna funkcia, ktorá vytvára inštanciu triedy. Volá sa rovnako ako trieda, ktorej prislúcha. 
Ak nie je špecifikovaná, volá sa v implicitnom konštruktore konštruktor nadtriedy. % máme niekde generatíve constructors? (this.id)
% factory constructors? constant constructors?

\subsection{Typy}
\label{subsec:typy}
Programovací jazyk Dart podporuje voliteľné typovanie založené na typoch rozhrania.

\paragraph{Statické typy}
Statické typy sú použité pri deklarovaní premenných, pri definícii návratových hodnôt funkcií a v ohraničení typu premenných. 
Tieto statické typy sú použité iba pri statickej kontrole a v kontrolovanom móde. Na produkčný mód nesmú mať žiaden vplyv.
Medzi základné typy patria:
\begin{itemize}
\item konštanty (\emph{constants}): čísla (\emph{number}), booleovské premenné (\emph{bool}), reťazce znakov (\emph{string}), nulový objekt (\emph{null})
\item kolekcie viacerých prvkov: zoznamy (objekt \emph{List}), mapy (objekt \emph{Map})
\end{itemize}

Pri produkčnom móde sa všetky typy nahradia jednotným typom \emph{dynamic}. Ten označuje neznámy typ.

\paragraph{Dynamic}
Typ \emph{dynamic} má definovanú každú možnú operáciu s všetkými možnými počtami parametrov. Návratová hodnota týchto operácií je vždy \emph{dynamic}. Chceme tak zabezpečiť, aby nám typ \emph{dynamic} nikdy nevrátil chybovú hlášku o nesprávnom type. \emph{Dynamic} je považovaný za typový objekt, aj keď to nie je trieda.

\paragraph{Void}
Typ \emph{void} možno použiť len ako návratovú hodnotu funkcie. \emph{Void} nie je považovaný za typový objekt. Môže oznamovať upozornenia v kontrolovanom móde, ak funkcia vracia inú hodnotu ako \emph{null}.

\paragraph{Null}
Rezervované slovo \emph{null} označuje null-ový objekt. Je to jediná inštancia triedy \emph{Null}. Rozširovanie, implementovanie alebo použitie tejto triedy ako mixin spôsobí chybu pri kompilácii (\emph{compile-time error}).
Volanie akejkoľvek metódy na objekte \emph{null} spôsobí chybu.
Statická verzia objektu null je $\bot$.

\paragraph{This}%velmi divotvorna konstrukcia
Slovo \emph{this} označuje aktuálne používanú inštanciu triedy, s ktorou pracujeme. Statický typ \emph{this} je potom rozhranie tejto triedy.
%Slovo \emph{this} označuje cieľ vyvolania aktuálnej inštancie. 
%Statický typ pre \emph{this} je rozhranie bezprostredne ohradenej triedy.

\subsection{Premenné}
Premenné sú úložiská v pamäti. Neinicializovaná premenná má hodnotu \emph{null}.

\emph{Static variable} je taká premenná, ktorá nie je asociovaná s konkrétnou inštanciou, ale s celou knižnicou alebo triedou.

\emph{Final variable} je taká premenná, ktorá je zviazaná s konkrétnym objektom od jej deklarácie. Spôsobuje \emph{static warning} ak je inicializovaná aj v konštruktore. Spôsobuje \emph{compile-time errror} ak nie je inicializovaná pri deklarácii.
\emph{Constant variable} je implicitne \emph{final}.
%opak final je mutable

Ak deklarácia nešpecifikuje typ premennej, tak je \emph{dynamic}, čo predstavuje neznámy typ.

\paragraph{Gettery a settery}
Ku premenným pristupujeme pomocou prirodzených getterov a setterov. 
Getter je funkcia bez argumentov, ktorá v čase zavolania vyhodnotí výraz, ktorý definuje danú premennú a vráti výsledok.
Setter je funkcia s jedným argumentom, ktorá danej premennej priradí hodnotu jej argumentu. \emph{Final} premenné nemajú settery.

\subsection{Funkcie}
Funkcie predstavujú vykonateľné akcie. Funkcie pozostávajú z deklarácií, metód, getterov, setterov, konštruktorov. % a literálov.
Každá funkcia má 2 časti: popis(signature) a telo(body). Popis funkcie obsahuje formálne parametre a môže obsahovať typ návratovej hodnoty. Telo funkcie môže mať 2 tvary:
\begin{itemize}
\item blok príkazov v zložených zátvorkách (\{, \}). Ak tento blok príkazov neobsahuje príkaz \emph{return}, automaticky sa na koniec pridáva s návratovou hodnotou \emph{null}.
\item \emph{=> e}, čo je ekvivalentné \emph{\{return e;\}}
\end{itemize}
Oba bloky príkazov môžu byť vykonané synchrónne(modifikátor \emph{sync*}) alebo asynchrónne(\emph{async, async*}).

Každá funkcia má zoznam parametrov, ktorý obsahuje zoznam povinných pozičných parametrov, potom zoznam voliteľných parametrov. Voliteľné parametre môžu byť pomenované alebo pozičné ale nie obe súčasne.

Ak sa neuvedie typ návratovej funkcie explicitne, jej typ je \emph{dynamic}, alebo daná trieda, ak ide o konštruktor.

Externá funkcia(\emph{external}) je funkcia, ktorá má deklaráciu a telo funkcie na rôznych miestach v kóde. Môžu to byť napríklad funkcie implementované externe v inom programovacom jazyku alebo také, ktoré sú dynamicky generované ale ich popis je statický a známy.

\subsection{Upozornenia a chyby}%warnings and errors
Dart rozlišuje niekoľko druhov chýb. Napríklad:
\begin{itemize}
\item Kompilačné chyby (\emph{compile-time errors}) teda chyby v čase kompilácie, ktoré bránia ďalšiemu behu programu. Tieto musia byť nahlásené kompilátorom pred spustením samotného chybného kódu.
\item Statické upozornenia (\emph{static warnings}) sú chyby zistené statickou kontrolou (teda nie za behu programu). Nemajú žiaden efekt v čase behu programu.
Statická kontrola sa týka najmä konzistentnosti typov, ale neznemožňuje kompiláciu ani beh samotného programu. Statickú kontrolu by mali zabezpečovať vývojové prostredia a kompilátory.
\end{itemize}

\paragraph{Módy behu programu}
Programy môžu byť spustené v dvoch módoch:%\paragraph{...} 
\begin{itemize}
\item Checked mode (\emph{kontrolovaný mód}) - v tomto móde fungujú \emph{static warnings} aj \emph{compile-time errors}. Je vhodný na písanie kódu a ladenie programu.
\item Production mode (\emph{produkčný mód}) - ako samotný názov napovedá, tento mód je určený na beh programu u klienta. V tomto móde sa \emph{static warnings} nevyskytujú, sú tu len \emph{compile-time errors} a chyby, ktoré sa vyskytli priamo pri behu aplikácie.
\end{itemize}

\subsection{Súkromie}
Dart podporuje dve úrovne súkromia, \emph{private} (súkromný) a \emph{public} (verejný).
Objekt, ktorý je deklarovaný s kľúčovým slovom \emph{private} je súkromný, inak je každý objekt verejný. 
Tiež môžeme definovať súkromný objekt tak, že jeho názov začína podčiarkovníkom (\glqq\_\grqq). 

Programy v jazyku Dart sú organizované do knižníc. Objekt m je dostupný v knižnici len ak je definovaný v danej knižnici, alebo ak je \emph{public}.

\subsection{Knižnice}
Program v jayzku Dart pozostáva z jednej alebo viacerých knižníc. Môže byť vytvorený z viacerých kompilačných jednotiek (\emph{compilation units}). Kompilačná jednotka môže byť knižnica alebo \emph{part}.
Knižnice sú jednotkami súkromia. Kód definovaný vrámci knižnice ako súkromný je dostupný len vrámci danej knižnice.

Knižnica pozostáva z množiny importov, exportov a verejne deklarovaných objektov. Tieto môžu byť triedy, funkcie alebo premenné.

\paragraph{Import}
Kľúčové slovo \emph{import} prepája knižnice. Určuje, ktoré knižnice môžu byť použité pri programovaní inej knižnice. 
\emph{Import} umožňuje explicitne povedať, ktoré objekty z kategórie \emph{public} chceme skryť, tie uvedieme za kľúčovým \emph{hide}. 
Alebo vybrať podmnožinu objektov, ktoré chceme ponechať (a ostatné sa nám skryjú) pomocou kľúčového slova \emph{show}.
Ak by sa nám mohlo stať, že názvy funkcií alebo premenných sa vo viacerých knižniciach prekrývajú (čo je problém a snažíme sa tomu zabrániť), môžeme premenovať dané objekty pomocou kľúčového slova \emph{as}. 
Podobne sa dá pomenovať aj samotná knižnica, kde potom voláme prvky knižnice nasledovne: \emph{libraryName.objectName}.

\paragraph{Export}
Kľúčové slovo \emph{export} definuje množinu objektov, ktoré sú prístupné po importovaní danej knižnice L, v ktorej sa \emph{export} nachádza. 
Môžeme exportovať množinu objektov, alebo celú knižnicu. 
Tiež môžeme obmedziť export knižnice pomocou \emph{show} a \emph{hide} rovnako, ako pri importovaní.

\paragraph{Parts}
Ak máme veľkú knižnicu, môžeme ju rozdeliť do viacerých súborov pomocou \emph{part} a \emph{part of}. 
Všetky definície objektov, aj súkromné, sú medzi týmito časťami vzájomne viditeľné.

Hlavný súbor obsahuje kľúčové \emph{part}, kde pomenuje cestu ku druhému súboru, ktorá reprezentuje časť knižnice. Tá pomenuje, ku ktorému hlavnému súboru prislúcha za kľúčovým \emph{part of}.
Importovanie ďalších knižníc potom stačí uviesť v hlavnom súbore.

\paragraph{Scripts}
\emph{Skript} sa nazýva knižnica, ktorá obsahuje funkciu \emph{main}. Takáto funkcia môže byť v jednom projekte práve jedna. Je to funkcia, ktorá sa spúšťa na začiatku programu.

\paragraph{Pub}
O to, aby boli všetky knižnice, ktoré sú importované, dostupné a aktualizované zabezpečuje špeciálny správca knižníc \emph{pub} (\emph{package manager}). 
Každá knižnica má zoznam závislostí - aké verzie cudzích knižníc používa. 
Pub vyžaduje špeciálny súbor (pubspec.yaml) obsahujúci zoznam knižníc s požadovanými verziami ku každej knižnici. Na základe tohoto súboru pracuje príkaz \emph{pub get}, ktorý stiahne potrebné zmeny.

Okrem iného spravuje \emph{pub} mená publikovaných knižníc, aby sa zabránilo kolíziám.
%TODO pub serve

\subsection{Súbežnosť}%paralelizmus 
Kód v jazyku Dart je vždy jednovláknový. Ak chceme vykonávať viac súbežných činností, používame špeciálnu entitu \emph{isolates}, ktorá má vlastnú pamäť a vlastnú kontrolu vlákna. Tieto entity medzi sebou komunikujú pomocou posielania správ, nezdieľajú žiaden stav.

Dart podporuje asynchrónnosť vykonávania programu. Kľúčovým \emph{await} odovzdáme kontrolu, pokým sa výraz za \emph{await} vyhodnotí. Ak sa nevie vyhodnotiť v čase vykonávania tejto inštrukcie, namiesto hodnoty sa vytvorí inštancia triedy \emph{Future}, za ktorú sa neskôr po dopočítaní dosadí daná hodnota výrazu.%neviem, či som to uplne dobre pochopila

%\subsection{TODO}
%kaskádová notácia %16.17.2

\section{\JS a jeho porovnanie s jazykom Dart}
Hlavné črty programovacieho jazyka \JS. 

ECMAScript bol pôvodne navrhnutý ako webový skriptovací jazyk, ktorý upravuje internetové stránky v prehliadači a vykonáva výpočty v prehliadači. Dnes je to plne vybavený všeobecne navrhnutý objektovo orientovaný programovací jazyk.
Skriptovací jazyk je programovací jazyk zameraný na výpočty a manipuláciu s objektami už existujúceho systému.
%ECMAScript je objektovo orientovaný programovací jazyk  bežiaci na hostiteľskom prostredí.
\TODO

\subsection{Triedy}
\TODO

\paragraph{Postrehy pri písaní}
\emph{Mixiny} som nahradila čistými funkciami, ktoré som dala do samostatného súboru a vytvorila tak malú knižnicu funkcií.
Nevýhodou pri používaní funkcií namiesto mixinov je nemožnosť využiť gettery. Všetky hodnoty, ktoré chceme použiť vrámci jednej funkcie musíme mať ako vstupné parametre funkcie.

\subsection{Typy}
Jazyk \JS nie je typovaný jazyk. Napriek tomu rozlišuje niekoľko základných typov, na ktorých má definované konkrétne operácie. Typy jazyka ECMAScript sú \emph{Undefined}, \emph{Null}, \emph{Boolean}, \emph{String}, \emph{Symbol}, \emph{Number} a \emph{Object}. Každá hodnota premennej v tomto jazyku je charakterizovaná jedným z uvedených typov.

\paragraph{Undefined} typ má práve jednu hodnotu \emph{undefined}. Každá premenná, ktorá nemá priradenú žiadnu hodnotu, má práve túto hodnotu.

\paragraph{Null} typ má práve jednu hodnotu, \emph{null}.

\paragraph{Boolean} typ reprezentuje logickú entitu, ktorá môže nadobúdať len dve hodnoty \emph{true} a \emph{false}.

\paragraph{String}
\TODO

\subsection{Premenné}
%Na typ konkrétnej premennej sa môžeme pýtať príkazom
\subsubsection{Deklarovanie premenných v \JS}
\JS podporuje primárne tri typy deklarovania premenných. Sú to \emph{var}, \emph{const} a \emph{let}. V prípade \emph{const} a \emph{let} ide o premenné, ktoré sa nedajú deklarovať dvakrát s rovnakým menom a platia len vrámci daného bloku kódu. V štandardnej terminológii by sme ich mohli nazvať aj lokálne premenné. Narozdiel od týchto, premennú \emph{var} môžeme deklarovať aj viackrát a môžeme sa na ňu pýtať aj mimo bloku kde sme ju deklarovali. V ukážke vidíme viacnásobné deklarovanie, volanie premennej mimo bloku, kde bola zavolaná aj volanie premennej predtým, ako bola vytvorená. Tieto premenné by sme mohli nazvať aj globálne premenné.

\begin{lstlisting}[caption=JavaScript deklarovanie]
  var i = 2;
  if (i >= 0) {
    console.log( "i =", i, ", j =", j ); 
    // output: i = 2 , j = undefined
    var j = 5;
    var i = 3;
    k = 10;
  } else {
    var j = 4;
  }
  console.log( "i =", i, ", j =", j, ", k =", k ); 
  // output: i = 3 , j = 5 , k = 10
\end{lstlisting}

%TODO funkcia deklarovaná ako konštanta z anonymnej funkcie

\subsection{Upozornenia a chyby}
\TODO

\subsection{Súkromie} ?? čo k tomu??
\TODO

\subsection{Knižnice}
\TODO import, export, npm

vs. pub

\subsubsection{Dostupnosť}
Dostupnosť jazykov Dart a \JS je rôzna. Zatiaľ čo jazyk JavaScript vie (takmer) každý prehliadač reprezentovať priamo, jazyk Dart je potrebné v produkcii prekladať do jazyka JavaScript (Pri vývoji aplikácie v Darte je možné použiť špeciálny prehliadač na tento jazyk).

Pre jazyk \JS je dostupné množstvo knižníc. Veľká časť z nich je dostupná cez správcu npm. Vkladanie knižnice do projektu je veľmi jednoduché volanie príkazu npm s vhodnou kombináciou prepínačov. (Volanie funkcie z priečinka projektu s prepínačom --save bolo u nás postačujúce.)

\subsection{Súbežnosť}
\TODO asynchrónnosť, niečo ako await?


\subsection{TODO}
\cite[introduction]{flanagan2006javascript} \cite[\JS]{ECMAScript}

\section{Spoločné znaky jazykov Dart a \JS}
Spoločné črty programovacích jazykov Dart a \JS. Pravdepodobne odkaz na tabuľku s podrobnejším popisom vybraných spoločných a rozdielných príkazov.

\section{Rozdiely a ich prepojenie}
Rozdielne časti daných jazykov a návrh riešenia, ako by sa dali nahradiť. Využitie zaujímavých čŕt jazyka \JS.
%(malé množstvo knižníc!!)

npm vs. pub