\section{Knižnice open source použité v aplikácii}%TODO preložiť? napisat Brejovej

\label{sec:kniznice}

\paragraph{Este}
%TODO boilerplate, a dalsie preklady
Celú aplikáciu sme začali vyvýjať v prostredí \cite[este]{Este}. Snaha držať sa lean prístupu vývoja aplikácie nás nasmerovala na využitie už existujúceho kódu. Táto sada knižníc je polotovar (\emph{boilerplate}) pre React-ové aplikácie. Jej poslaním je priniesť rozbiehajúcim sa projektom minimálny životaschopný produkt tak rýchlo, ako je to možné. V aktuálnej verzii podporuje vývoj aplikácie s návrhovým vzorom Redux. Zo zaujímavých knižníc obsahuje react-intl na správu jazyka, redux-observable pre vedľajšie efekty a nástroj gulp pre jednoduché spúšťanie bežiaceho projektu.

\paragraph{React}
React \cite{react} je deklaratívna, efektívna a flexibilná JavaScript-ová knižnica na vytváranie používateľského rozhrania. Vytvára komponenty, ktoré sa vykreslia na základe argumentov (\emph{props}) a vnútorného stavu, ktorý si môžu ukladať. 

Na vykreslenie komponentov sme pri vývoji použili túto knižnicu. Jej veľkou výhodou je, že je rozšírená medzi programátormi a existuje pre ňu mnoho ďalších kompatibilných knižníc. Tiež veľmi pekne spolupracuje s našim návrhovým vzorom \emph{Redux}, keďže React-ové komponenty majú úlohu dáta vykresliť.

V našej aplikácii používame krátky zápis pre vytvorenie takéhoto komponentu. Tiež sa snažíme nepoužívať vnútorný stav aplikácie. Všetky dáta, od ktorých závisí vykreslenie, by mali pochádzať zo stavu aplikácie. Ak pridáme komponentu vhodný kľúč, knižnica vie menežovať, či daný komponent treba prekresliť pri zmene stavu aplikácie.

Pri práci s knižnicou \emph{react} sme zaznamenali nevýhodu, keď nebolo možné používať syntax pre jeden komponent na pole viacerých komponentov, čo sme vyriešili obalením poľa pomocným komponentom. Tiež by sa tento problém dal riešiť použitím funkcie namiesto syntaxe pre komponent.

%TODO skontrolovat pred tlacenim ramiky listingov
\begin{lstlisting}[caption=Pole komponentov v Dart-e]
  get buttons => [
    proforma,
    packed,
    cancelOrder,
    paid,
  ];
\end{lstlisting}

\begin{lstlisting}[caption=Pole komponentov v JavaScripte s použitím knižnice React]
const Buttons = order => (
  <Box>
    <Proforma props={{ order }} />
    <Packed props={{ order }} />
    <CancelOrder props={{ order }} />
    <Paid props={{ order }} />
  </Box>
);
\end{lstlisting}

\paragraph{react-native}
React native \cite{ReactNative} je knižnica, ktorá umožňuje vytvárať natívne aplikácie pre mobilné zariadenia. Využíva syntax knižnice Redux a zároveň na zariadení pracuje priamo s objektami zariadenia pri vykreslení komponentov. Pre túto knižnicu funguje hot-reloading.

\paragraph{redux}
\cite[Redux]{redux-lib} je knižnica, ktorá pomáha vytvárať Redux-ové aplikácie pre prehliadače, mobilné zariadenia aj server. Poskytuje nástroje pre správu častí reducer, store, middleware z návrhového vzoru Redux. 

Pre správu komponentov v knižnici React existuje pre knižnicu \emph{redux} ďalšia knižnica \emph{react-redux} \cite{react-redux}, ktorá prepája komponenty so stavom.

\subparagraph{Connect}
\label{func:connect}
Funkcia \emph{connect} z knižnice \emph{react-redux} \cite{react-redux} prepája stav s komponentami. Táto funkcia akceptuje dva argumenty. Jedným z nich je funkcia, ktorá namapuje stav na argumenty komponentu. Druhý je funkcia, ktorá namapuje dispatcher na akcie. %voliteľné - null
(\emph{Connect} môžeme použiť aj na zmenu vlastných parametrov pre prezenčné komponenty, avšak to väčšinou nie je žiadúce, pretože to komplikuje logiku kódu. Do prezenčných komponentov by sme mali posúvať už len hotové dáta na vykreslenie. Predídeme tak aj zbytočnému prekresľovaniu kódu kôli zmeneným vlastným dátam.)
%TODO ukážka connect funkcie


%funkciu na vyskladanie jedného reducera z veľa menších (\emph{combineReducers}), vytvorenie store z reducera (\emph{createStore}), použitie middlewares (\emph{applyMiddleware}), alebo ďalšie.

\paragraph{Router}%TODO preložiť? smerovanie
O niečo zložitejšie je routovanie a správa url v aplikácii. Existuje viacero možností, ako riešiť routovanie. 

Zmenu routy riešime pomocou knižnice \emph{react-router}. %link
Idea tejto knižnice je založená na komponente \emph{Link}, ktorý vytvorí akciu na zmenu routy. Úlohou tohoto komponentu je okrem iného zistiť, na akej route sa nachádza. 
Pri vykreslení aplikácie komponent \emph{Route} rozhoduje, ktorý komponent bude vykreslený. % v (routeConfig.js)
V čase, keď vybraný zavolá, mu do argumentov (props)
dá aj routu na ktorej sa nachádza. Z nej potom možno funkciou \emph{connect} zistiť potrebné informácie o aktuálnej route. %TODO ukážka OrderPage

Jej výhodou je, že routovanie z komponentov je veľmi jednoduché a prirodzené. Čo nám chýbalo, bola málo popísaná možnosť meniť adresu mimo komponentov. Túto vlastnosť by sme veľmi ocenili najmä kôli ideológii Redux-u, keďže by mal byť jediným zdojom pravdy práve stav aplikácie. Po použití tejto knižnice máme zdroje pravdy aspoň dva, jeden pre dáta aplikácie a druhý pre adresu url. 


Trošku \uv{krajšie} v zmysle Redux-ovej logiky by bolo riešenie s použitím knižnice \emph{router-5} alebo \emph{react-router-redux}, ktorá rieši celý routing na základe dát v stave, kam si ukladá informácie o aktuálnej adrese (aj predchádzajúcich).

\paragraph{Ďalšie knižnice}
\begin{itemize}
  \item gulp \cite{gulp} - Nástroj na automatizáciu vývojových úloh, ako napríklad vytvorenie bežiaceho prostredia v prehliadači na vývoj aplikácie.
  \item react-intl \cite{intl} - Táto knižnica poskytuje React-ové komponenty a API na formátovanie času, čísiel a textov vrátane množných čísel a spravovanie prekladov.
  \item material-ui \cite{material-ui} - Knižnica React-ových komponentov v material dizajne najmä pre mobilné telefóny. %react-toolbox?
  \item normalizr \cite{normalizr} - Knižnica normalizuje dáta vo formáte JSON podľa zadanej schémy. Vráti pole identifikátorov a mapu normalizovaných objektov.
  \item webpack \cite{webpack} - Knižnica slúži na previazanie modulov aplikácie v JavaScript-e do malého počtu súborov, ktoré sa ľahko importujú na stránku.
\end{itemize}